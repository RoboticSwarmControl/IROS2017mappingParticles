%%%%%%%%%%%%%%%%%%%%%%%%%%%%%%%%%%%%%%%%%%%%%%%%%%%%%%%%%%%
\section{Conclusion and Future Work}\label{sec:conclusion}
%%%%%%%%%%%%%%%%%%%%%%%%%%%%%%%%%%%%%%%%%%%%%%%%%%%%%%%%%%%

This paper presented techniques for controlling particle swarms in 1D and 2D grids.
These particles can be tracked and controlled by an external agent, but control inputs are applied uniformly so that each particle experiences the same applied forces. 

We provided theoretical and practical insights  with potential relevance for fast MRI scans with magnetically controlled contrast media.
In particular, we developed an approach for searching for an object at an unknown distance $D$, where the search is subject to two different and independent cost parameters
for {\em moving} and for {\em measuring}. 
We showed that regardless of the relative cost of these two operations,
there is a simple $O(\log D/\log\log D)$-competitive strategy. Extending the 1D bicriterion to an arbitrary freespace polyomino is not straightforward,
as a two-dimensional scenario has to deal with more intricate topological and geometric difficulties. This leaves the analytic treatment as a future challenge.
This paper also presented benchmark algorithms for 2D mapping and coverage problems.
These results form a baseline for future work, which should focus on improving performance. 
Extensions to 3D and continuous spaces are especially relevant to the motivating problem of MR-scanning in living tissue.